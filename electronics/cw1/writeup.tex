\documentclass[a4paper,12pt]{article}

\usepackage{geometry}
\usepackage{titling}
\usepackage{titlesec}
\usepackage[english]{babel}
\usepackage[hidelinks]{hyperref}
\usepackage{listings}
\usepackage{xcolor}
\usepackage{graphicx}
\usepackage{forest}
\usepackage{tikz-qtree}
\usepackage{setspace}


\titleformat{\section}
{\Huge}
{}
{0em}
{}[\titlerule]
\geometry{a4paper,total={170mm,257mm},left=25mm,right=25mm,}

\author{Lucas Standen}
\title{Creating a simple temprature sensing circuit}



\begin{document}
\maketitle

\newpage

\tableofcontents
\newpage

\setlength{\parskip}{1em}

{\setlength{\parindent}{0cm}

\section{System Planning}
\subsection{Problem analysis}
My circuit will sense temperature, and will be taking into consideration pet owners, worried about 
their homes over-heating for their pets, this will be especially helpful for owners of sensitive pets 
such as fish. People who own these pets often leave them at home alone, which can be deadly on summer 
days, my device plans to alert the owner, and can be attached to other systems such as a cooling system. 

My system, will flash an LED and pulse a buzzer to make it clear that it is too hot, have an indicator 
to tell the user that something has gone wrong, and have a pin to free to attach to an external system.
It will have a adjustment dial to change the threshold, so the user can specify what temperature is too
hot.

\subsection{Who is it for?}
\subsection{Design specification}

\section{System Design}
\subsection{Showing how it will function}
\subsection{The code}

\section{System Realisation}
\subsection{Ciruit realisation}
\subsection{Calibrating the sensors}
\subsection{Results}

\section{System Evaluation}
\subsection{Did it work?}
\subsection{What could go better?}
}
\end{document}
